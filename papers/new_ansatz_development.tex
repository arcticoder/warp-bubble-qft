\section{Metric Ansatz Exploration}

\subsection{General Variational Principle}
To minimize the total negative energy
\[
  E_{-}[f] \;=\;\int_0^R \rho_{\rm eff}\bigl(f(r),\,f'(r)\bigr)\;4\pi\,r^2\,dr,
\]
we solve
\[
  \frac{\delta E_{-}}{\delta f(r)} = 0,
\]
subject to $f(0)=1$, $f(R)=0$, and smoothness conditions.

\subsection{Polynomial Ansatz}
Let
\[
  f(r; a_0,\,a_1,\dots,\,a_N) 
  = 
  \begin{cases}
    1 & 0 \le r \le r_0, \\
    \displaystyle 
    \sum_{k=0}^N a_k \bigl(\tfrac{r - r_0}{R - r_0}\bigr)^k 
      & r_0 < r < R, \\
    0 & r \ge R,
  \end{cases}
\]
with $a_0=1$, $a_N=0$.  Enforce $\partial_r f|_{r_0}=\partial_r f|_{R}=0$.  
The Euler–Lagrange equation gives
\[
  \frac{d}{dr}\Bigl( r^2\,\frac{\partial \rho_{\rm eff}}{\partial f'}\Bigr) 
    - r^2\,\frac{\partial \rho_{\rm eff}}{\partial f} = 0.
\]
Insert $\rho_{\rm eff}(r)= -\tfrac{v^2}{8\pi}\,\beta_{\rm backreaction}\,\sinc(\pi\mu)\,(f')^2 / \mathcal{R}_{\rm geo}$,
then project onto the polynomial basis to get a linear system for $\{a_k\}$.

\subsection{Exponential Ansatz}
A simpler two‐parameter family:
\[
  f(r;\alpha,\beta) 
  = 
  \begin{cases}
    1 & r \le r_0, \\
    \exp\bigl[-\alpha\,\frac{r-r_0}{R-r_0}\bigr] & r_0 < r < R, \\
    0 & r \ge R.
  \end{cases}
\]
With continuity at $r=r_0$ and $r=R$, determine $\alpha$ by minimizing
\[
  E_{-}(\alpha) = \int_0^R 
    \Bigl[-\tfrac{v^2}{8\pi}\,\beta_{\rm backreaction}\,\sinc(\pi\mu)\,(f'(r;\alpha))^2 / \mathcal{R}_{\rm geo}\Bigr]\;4\pi\,r^2\,dr.
\]
Set $dE_{-}/d\alpha = 0$ to solve for $\alpha$.

\subsection{Soliton‐Like (Lentz) Ansatz}
Following Lentz (2019),
\[
  f(r) = \sum_{i=1}^M A_i \,\sech^2\!\Bigl(\tfrac{r - r_{0,i}}{\sigma_i}\Bigr),
\]
with parameters $\{A_i,\;r_{0,i},\;\sigma_i\}$.  Again enforce $f(0)=1$, $f(R)=0$, and apply
\[
  \frac{\partial E_{-}}{\partial A_i} = 0,\quad
  \frac{\partial E_{-}}{\partial r_{0,i}} = 0,\quad
  \frac{\partial E_{-}}{\partial \sigma_i} = 0,
\]
to find the global optimum.

\subsection{Lentz‐Gaussian Superposition}
Another family:
\[
  f(r) = \sum_{i=1}^M B_i\,\exp\!\Bigl[-\tfrac{(r - r_{0,i})^2}{2\sigma_i^2}\Bigr],
\]
etc.  The same variational principle applies.

\subsection{8-Gaussian Two-Stage Ansatz}
The latest breakthrough in ansatz development utilizes an 8-Gaussian two-stage optimization approach:
\[
  f(r) = \sum_{i=1}^8 A_i\,\exp\!\Bigl[-\tfrac{(r - r_{0,i})^2}{2\sigma_i^2}\Bigr],
\]
with sophisticated parameter initialization and staged optimization. The two-stage process consists of:

\subsubsection{Stage 1: Coarse Grid Exploration}
Initial parameter sweep using lower resolution (N=400) with Differential Evolution:
\begin{itemize}
\item Population size: 16
\item Maximum iterations: 100  
\item Parallel workers: all available cores
\item Parameter bounds: $\mu \in [10^{-8}, 10^{-4}]$, $\mathcal{R}_{\text{geo}} \in [10^{-6}, 10^{-3}]$
\end{itemize}

\subsubsection{Stage 2: High-Resolution Refinement}
Fine optimization on promising candidates using higher resolution (N=800):
\begin{itemize}
\item CMA-ES optimizer with population size 24
\item Maximum iterations: 200
\item L-BFGS-B polishing step
\item Enhanced physics constraints and penalty functions
\end{itemize}

This approach achieves record-breaking negative energy densities:
\[
  E_- \approx -2.35\times10^{31}\,\text{J}
  \quad(\mu \approx 3.2\times10^{-6},\,\mathcal{R}_{\rm geo} \approx 1.8\times10^{-5}).
\]

The 8-Gaussian ansatz provides superior flexibility for capturing complex bubble wall structures while maintaining computational efficiency through the two-stage optimization strategy.

\subsection{Hybrid Spline-Gaussian Method}
An alternative high-dimensional approach combines spline interpolation with Gaussian components:
\[
  f(r) = 
  \begin{cases}
    1, & 0 \le r \le r_0,\\
    S_{\text{spline}}(r), & r_0 < r < r_{\text{transition}},\\
    \sum_{i=1}^{N_G} C_i\,\exp\!\Bigl[-\tfrac{(r - r_{0,i})^2}{2\sigma_i^2}\Bigr], & r_{\text{transition}} \le r < R,\\
    0, & r \ge R.
  \end{cases}
\]

Key parameters:
\begin{itemize}
\item Spline order: cubic (k=3)
\item Number of spline knots: 12-16
\item Gaussian components: 4-6
\item Transition point optimization: variational principle
\end{itemize}

This hybrid method targets energy densities approaching $E_- \sim -2.5\times10^{31}$ J through enhanced wall profile flexibility, though at increased computational cost compared to pure Gaussian ansätze.

\subsection{Implementation Notes}
These ansätze are implemented in \texttt{MetricAnsatzBuilder} and optimized by \texttt{MetricAnsatzOptimizer} (see \texttt{optimize.py}).
